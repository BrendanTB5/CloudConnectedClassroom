\subsection*{Gateway}

\subsubsection*{How to Compile}

To compile the program, you need to open the following file in M\+P\+L\+A\+BX.

\begin{quote}
Gateway/demo/microchip/pic32mz\+\_\+ef\+\_\+curiosity/wifi\+\_\+http\+\_\+server\+\_\+demo/mplabx \end{quote}


This file has been proved to compile in M\+P\+L\+A\+BX v5.\+3.\+0 using the X\+C32 compiler v2.\+30

\subsubsection*{How to Run}

This program requires the first generation P\+I\+C32 MZ EF Curiosity. The board utilizes an P\+I\+C32\+M\+Z2048\+E\+F\+M100 microprocessor. This has not been verified to run on the second generation of the board, and since it uses a different layout and microprocessor will not run most likely. The software requires the left mikrobus slot to be populated by the Wi-\/\+Fi 7 click module, the right mikrobus slot to be populated with the U\+A\+RT to U\+SB Click, and the Ethernet adapter to be populated.

\subsubsection*{How to Connect to A\+WS}

The current program requires the A\+WS gateway database to populated with the device\textquotesingle{}s M\+AC address. This can be found by anyone with access to D\+H\+CP logs for the local network. The Client ID must be set to a A\+WS device name, and the client certificate must match one connected to that Client ID. Ensure that the endpoint url is changed to the given IoT Core endpoint url.

\subsubsection*{How to Read Documentation}

The project is documented using Doxygen. Click on the link to index.\+html, if that is not available open the file at

\begin{quote}
Gateway/\+Documentation/html/index.\+html \end{quote}


Most of the documentation that is needed for our changes is in main.\+c This can be searched for, and it will give in depth descriptions of the code we have made. Hope you have as much fun with this as I did. 